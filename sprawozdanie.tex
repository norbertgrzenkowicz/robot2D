\documentclass{article}

% Language setting
% Replace `english' with e.g. `spanish' to change the document language
\usepackage[polish]{babel}
\usepackage{listings}
% Set page size and margins
% Replace `letterpaper' with `a4paper' for UK/EU standard size
\usepackage[letterpaper,top=2cm,bottom=2cm,left=3cm,right=3cm,marginparwidth=1.75cm]{geometry}
\usepackage{array}
% Useful packages
\usepackage{amsmath}
\usepackage{graphicx}
\usepackage[colorlinks=true, allcolors=blue]{hyperref}
\usepackage{tikz}
\usepackage{listings}
\usepackage{xcolor}

\definecolor{codegreen}{rgb}{0,0.6,0}
\definecolor{codegray}{rgb}{0.5,0.5,0.5}
\definecolor{codepurple}{rgb}{0.58,0,0.82}
\definecolor{backcolour}{rgb}{0.95,0.95,0.92}

\lstdefinestyle{mystyle}{
    backgroundcolor=\color{backcolour},   
    commentstyle=\color{codegreen},
    keywordstyle=\color{magenta},
    numberstyle=\tiny\color{codegray},
    stringstyle=\color{codepurple},
    basicstyle=\ttfamily\footnotesize,
    breakatwhitespace=false,         
    breaklines=true,                 
    captionpos=b,                    
    keepspaces=true,                 
    numbers=left,                    
    numbersep=5pt,                  
    showspaces=false,                
    showstringspaces=false,
    showtabs=false,                  
    tabsize=2
}

\lstset{style=mystyle}

\title{Projekt Robot 2D}
\author{Stanisław Białecki, Andrzej Datta, Norbert Grzenkowicz}

\begin{document}
\maketitle



\section{Cel ćwiczenia}

Celem ćwiczenia było stworzenie symulacji działania robota 2D.

\section{Obliczenia}

\subsection{Kinematyka prosta}

\begin{center}
    \begin{figure}[!ht]
\centering
\includegraphics[width=0.3\textwidth]{prosta.png}
\caption{\label{fig:prosta}Schemat kinematyki prostej}
\end{figure}
\end{center}



\begin{table}[!ht]
\centering
\begin{tabular}{|l|l|l|l|1|}\hline
  &  $a_i$  &  $\alpha_i$ & $d_i$ & \phi \\\hline
 1 & $a_1 = L_1$  & 0 & 0 & $d_1$ \\\hline
 2 & $a_2 = L_2$ & 0 & 0 & $d_2$\\\hline
\end{tabular}
\caption{Notacja D-H}
\end{table}

Wzory potrzebne przy wyznaczaniu pozycji maniupulatora:


\begin{center}
    $x_1 = L_1\cos{\phi_1}$\\
    $y_1 = L_1\sin{\phi_1}$\\
    \linebreak
    $x = L_1\cos{\phi_1} + L_2\cos{\phi_1 + \phi_2}$\\
    $y = L_1\sin{\phi_1} + L_2\sin{\phi_1 + \phi_2}$\\
\end{center}

\newpage

\subsection{Kinematyka odwrotna}

\begin{center}
    \begin{figure}[!ht]
\centering
\includegraphics[width=0.3\textwidth]{prosta.png}
\caption{\label{fig:prosta}Schemat kinematyki odwrotnej}
\end{figure}
\end{center}

Potrzebne wzory

\begin{center}
    $\phi_1 = \arctan\frac{y}{x} - \arccos\frac{L^2_1+L^2-L^2_2}{2L_1L_2}$
    $\phi_2 = \pi - \arccos\frac{L^2_1+L^2-L^2_2}{2L_1L_2}$
\end{center}


\tikzset{every picture/.style={line width=0.75pt}} %set default line width to 0.75pt        

\begin{tikzpicture}[x=0.75pt,y=0.75pt,yscale=-1,xscale=1]
%uncomment if require: \path (0,300); %set diagram left start at 0, and has height of 300

%Shape: Axis 2D [id:dp9941783090617922] 
\draw  (0,271.7) -- (657,271.7)(65.7,-1) -- (65.7,302) (650,266.7) -- (657,271.7) -- (650,276.7) (60.7,6) -- (65.7,-1) -- (70.7,6)  ;
%Straight Lines [id:da7352810510562711] 
\draw [color={rgb, 255:red, 65; green, 117; blue, 5 }  ,draw opacity=1 ]   (211,87) -- (311,187) ;
%Straight Lines [id:da20860164752486132] 
\draw [color={rgb, 255:red, 245; green, 166; blue, 35 }  ,draw opacity=1 ]   (380,119) -- (311,187) ;




\end{tikzpicture}



\newpage
\subsection{Python}

\lstinputlisting[language=Python]{robot2d.py}

\section{Podsumowanie}
Zadanie zostało zrealizowane zgodnie z  \href{https://enauczanie.pg.edu.pl/moodle/mod/assign/view.php?id=1266783}{wymaganiami}. Symulacja robota 2d odbywa się w czasie ciągłym.



\bibliographystyle{alpha}

\begin{thebibliography}{9}
\bibitem{tematy}
Dr inż. Krzysztof Armiński, \emph{Tematy}


\end{thebibliography}

\end{document}